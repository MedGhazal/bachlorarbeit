\documentclass[
	12pt, 
	a4paper, 
	%draft
]{article}
\usepackage{geometry}
\geometry{left=1.6cm, right=1.6cm, bottom=3cm, top=3cm}

\usepackage{blindtext}
\usepackage{amsmath}
\usepackage{float}
\usepackage{multicol}
\usepackage{amsmath}
\usepackage{graphicx}
\usepackage{xcolor}
\usepackage[breaklinks]{hyperref}
\usepackage{microtype}
\usepackage[backend=biber]{biblatex}
\addbibresource{bibliography.bib}
\usepackage[T1]{fontenc}
\usepackage{fourier}
\usepackage{baskervald}
\hfuzz=8.002pt
\vfuzz=8.002pt
\hbadness=99999

\author{Mohamed Ghazal}
\title{Game engines}
\date{\today}

\begin{document}
	\maketitle
	In this document an overview of many 3D physics engines will be made. This will facilitate the arduous process of choosing a suitable 3D physics engine for our application. This is completely dependent of the efficiency of the implementation and its openness, and the following criterion:
	\begin{itemize}
		\item Usability.
		\item Technical support and documentation.
		\item Performance.
		\item Language: the language used to implement and extend the engine.
		\item Portability: the supported platforms.
		\item Licensing.
	\end{itemize}
	\section{Unreal}
	Unreal engine 4 uses the C++ as a programming language, but working with it doesn't require C++. Instead the developer Unreal engine 4 uses its scripting system and blueprint\cite{hamalainen2020game}. To write these scripts and use blueprints one can install plugins to many existing editors and suitable IDEs, such as Unreal Engine Tools by Tiago Patrocinio for VS Code. Unreal Engine 4 being open source offers a more flexibility in the process of development, in comparison with the Unity 3D engine, which offers open source only for users with a professional licence\cite{christopoulou2017overview}.\newline
	Just like the Unity 3d engine, unreal has its own programming editor Kismet to streamline the development process. In addition to that, Unreal engine supports importing/exporting content from/to Maya, 3D Studio MAX and blender and includes all possible developer toolkits such as Windows Phone SDK, iOS SDK, Android SDK and Android GDK. Furthermore Unreal Engine offers free technical support. Lastly, Unreal Engine 4 cannot be installed on Mac Os and Linux, and only compatible with windows systems. Thus, making the portability of Unreal engine projects into other systems partially questionable\cite{christopoulou2017overview}.\newline
	As for the development environments, Unreal engine 4 projects can be developed in Mac Os, Linux and windows. According to Eleftheria Christopoulou and Stelio Xinogalos, Unreal engine 4 has a more complex interface having an editor with many windows in comparison with Unity 3D. Furthermore, this engine uses a Blueprint Visual Scripting system that is based on node-based interface. This is a graphical editor for defining objects and classes, which can be very useful for non-programmers. However, Unreal engine is more suited for experienced users, as it support the visual scripting system, thus having a very complex graphical environment with a steep learning curve. Moreover, Unreal engine requires high performing hardware :q\cite{christopoulou2017overview}. Finally, it can be said that Unreal engine has a wide community offering a lot of tutorials and help for newbies and veterans alike.
	\section{Bullet}
	\section{Blender}
	Blender is a 3D graphics software used for animation, visualization and modeling, and it has a real-time 3D viewer and GUI. In addtion to that, blender has a scriptable interface to import/export data from/to it\cite{kent20153d}. Ruzinoor Che Mat et al. describe blender as (mesh) modelling and animation software, that comes with a \cite{lind2012using}
	\section{Unity}
	Unity 3D is created by Unity Technologies and it can be used to create both online and offline games. This physics engine is well documented and offers a variety of online tutorials for better entry opportunities. It utilizes the concept of assets, which can be copied as files in the source directory or imported directly. This versatility allows the developers to gain much control over the assets, that they are using. Their modular design is one of the main advantages of this physics engine. In addition to that, it is very portable, and works on many platforms, such as Mac os, Windows, Linux and mobile.\cite{mat2014using}\newline
	As for the usability of the engine, it has a dedicated editor MonoDevelop and a GUI interface, and it was implemented with C\#, Boo and JavaScript.\cite{christopoulou2017overview} The later makes Unity 3D compatible for the usage in Web Development to implement Website incorporating rudimentary animations, physically accurate simulations. Furthermore, the game engine is with supports importing and exporting content from/to the best known CAD-Platforms, such as Blender, Maya and 3D Studio Max.\cite{christopoulou2017overview}.\newline
	Eleftheria Christopoulou and Stelios Xinogalos in their comparative analysis of game engines ranked Unity 3D as the most usable game engines by including a lot of tutorial, examples and assets. However, their source remain closed for users of their personal version. Furthermore, the community of Unity 3D is very large giving access to more more online support for new users\cite{christopoulou2017overview}. Eleftheria Christopoulou and Stelios Xinogalos consider Unity 3D a suitable game engine for beginners for having a simpler and refined user interface and providing a lot of resources for them. In addition to that, the hardware requirements aren't too steep and they see the development and export of mobile games with Unity 3D as very straightforward and easy\cite{christopoulou2017overview}.
	\section{pychrono}
	\section{Opensim}
	\section{Panda 3D}
	\printbibliography
\end{document}
