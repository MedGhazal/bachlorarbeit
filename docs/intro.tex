\chapter{Introduction}
	\pagenumbering{arabic}
	In this thesis, the theme of Activity Recognition(HAR) will be discussed at length, and some venues of development will be touched upon. In the literature, it is understood that activity recognition is the discipline in the domain of machine learning, in which human activities are quantified and classified\cite{adeli2014multi}. To this end, many applications were developed that were based on many input data, a clear example of that is video input\cite{adeli2014multi}. In addition to that other metrics can be used to perform the core tasks of activity recognition such as:
	\begin{itemize}
		\item surveillance and monitoring, such as tracking sports activities and contributing for example to competitive and recreational sports.
		\item human-computer interaction for assistive and complementary functionalities\cite{adeli2014multi}.
	\end{itemize}
	This was facilitated by the wealth of new devices that were developed in the last 20 years, such as wearables, in which some very capable sensors and systems are embedded. An example of that is smart watches, e.g. Apple Watches and Garmin Forrunner, that use only information about the hand such as the velocity and acceleration to perform HAR. The spectrum of systems that are used in this context is wide, and on the other end of it, there are complete systems proving more accurate reading for scientific purposes\cite{6365160}. In this thesis, the KIT Motion-Language Dataset will be used to assemble a dataset based on the information stored in the xml-files that represent the motion in the mmm-notation. This dataset was developed by the KIT to provide a novel way of representing and collecting motion-data. Here the researchers tried to combine human motions and natural language, as this kind of dataset didn't not exist\cite{Plappert2016} by capturing motions and human activities but using a novel way of labeling, where each motion is annotated with multiple phrases having the same meaning.\newline
	In this thesis, a learning machine is the machine leaning algorithm used to model or learn features in a dataset, and this term will be used interchangeably with model and machine learning algorithm. In chapter \ref{chap:theoretical_foundations} the theoretical foundations are laid, such as the base of HAR and the methods used to create learning machines capable of extracting features to correctly identify and classify human activities. In addition to that, the use of deep neural networks and their potential will be discussed at length. First, the one-to-one models will be discussed in the context of HAR and recurrent neural networks as a special class of deep neural networks more suitable learning machines for the task of time series classification.\newline
	Next, the implementation of a framework for extracting and testing leaning machines will be discussed in chapter \ref{chap:project}, where I worked on three axes. First, the parsing and extraction of the data stored in the xml-files and concurrently the use of basic information retrieval techniques to label each motion. Next, the use of \textit{pyTorch} to implement the learning algorithms. The last axis is to implement a module to allow for rapid prototyping.\newline
	In the last chapter of this thesis, the evaluation of the methods used will be performed based on the results obtained by the implemented learning machines. The goal is to investigate the claim that recurrent neural networks are more suitable for the task of HAR. Intuitively, these learning machines are the best fit, but the worth of the additional computational overhead is very questionable. Nevertheless, these neural networks proved to be very hard to train due to their inherent problems, such as vanishing/exploding gradients, but the advent of new variants, e.g. LSTM and GRU, sparked new interest in these models and helped researchers overcome these problems.
