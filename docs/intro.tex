\chapter{Introduction}
\pagenumbering{arabic}
In this document the theme of Activity Recognition(HAR) will be discussed at length, and some venues of development will touched upon. In the literature it is understood that, activity recognition is the discipline in the domain of machine learning, in which the human activities are quantified and classified. To this end, many applications were developed that were based on many input data, a clear example of that is video input\cite{adeli2014multi}. In addition to that other metrics can be used to perform the core tasks of activity recognition such as:
\begin{itemize}
	\item surveillance and monitoring, such as tracking sport activities and contributing for example to competitive and recreational sports.
	\item human computer interaction for assistive and complementary functionalities\cite{adeli2014multi}.
\end{itemize}
This was facilitated by the wealth of new devices that were developed in the last 20 years, such as wearables, in which some very capable sensors and systems are embedded. An example of that are smart watches, e.g. Apple Watches and Garmin Forrunner, that use only ... to perform HAR. The spectrum of systems that are used in this context is wide, and on the other end of it, there are complete systems proving more accurate reading for scientific purposes\cite{6365160}. In this document the KIT Motion-Language Dataset will be used, developed by the KIT to provide a novel way in representing and collecting motion-data. Here the researchers tried to combine human motions and natural language, as such a dataset didn't not exist\cite{Plappert2016}.\newline
The aim of this bachelor thesis I will discuss the theoretical foundation of motion data, and how they could be processed in a machine learning to perform HAR in a practical sense. Using a physics simulation provided to us I will simulate the motion and extract more information to provide more context to the learning machine. To gauge the performance of transformers, using other learning machines such as CNNs, RNNs or LSTM-Nets as base lines would be pivotal. This study we be performed after laying a theoretical foundation to the motion data, that are XML-Data, the learning machines themselves and the HAR Problem itself. At the end of this study I want to outline, the added benefit of taking more information into account, such as contact forces with the floor and torque forces at the joints.  
