\section*{Abstract}
	In this thesis, the KIT motion dataset\cite{Plappert2016} will be used to evaluate learning machines for the purpose of human activity recognition. Whereby, some model sequences better than others and provide a better theoretical basis for this premise. Nevertheless, vanilla learning machines that build a one-to-one relationship between the data and the labels can perform better than recurrent neural networks, which are specially devised for sequence modeling. This thesis will investigate these assumptions using a suite of vanilla and sequence modeling neural networks by gauging their performance in a comparative study. To facilitate that, a small framework was built to experiment with and evaluate these learning machines.
\section*{Zusammenfassung}
	In dieser Arbeit wird der KIT-Bewegungsdatensatz\cite{Plappert2016} verwendet, um maschinelles Lernen Algorithmen für den Zweck der menschlichen Aktivitätserkennung zu evaluieren. Wobei einige Modelle Sequenzen besser modellieren als andere und eine bessere theoretische Grundlage bieten, warum das der Fall ist. Vanilla Learning Machines, die eine Eins-zu-Eins-Beziehung zwischen den Daten und den Labels bilden, können besser leisten als rekurrente neuronale Netze, die speziell für die Sequenzmodellierung modelliert sind. In dieser Arbeit wird dies anhand einer Reihe von Vanilla- und Sequenzmodellierungs-Neuronalnetzen untersucht und ihre Leistung in einer vergleichenden Studie bewertet. Um dies zu erleichtern, wurde ein kleines Framework erstellt, mit dem die Daten aus dem KIT Bewegungsdatensatz extrahiert und die Leaning Machines getestet und bewertet werden können. 
